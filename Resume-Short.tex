%%%%%%%%%%%%%%%%%%%%%%%%%%%%%%%%%%%%%%%%%
% Plasmati Graduate CV
% LaTeX Template
% Version 1.0 (24/3/13)
%
% This template has been downloaded from:
% http://www.LaTeXTemplates.com
%
% Original author:
% Alessandro Plasmati (alessandro.plasmati@gmail.com)
%
% License:
% CC BY-NC-SA 3.0 (http://creativecommons.org/licenses/by-nc-sa/3.0/)
%
% Important note:
% This template needs to be compiled with XeLaTeX.
% The main document font is called Fontin and can be downloaded for free
% from here: http://www.exljbris.com/fontin.html
%
%%%%%%%%%%%%%%%%%%%%%%%%%%%%%%%%%%%%%%%%%

%----------------------------------------------------------------------------------------
%	PACKAGES AND OTHER DOCUMENT CONFIGURATIONS
%----------------------------------------------------------------------------------------

\documentclass[a4paper,10pt]{article} % Default font size and paper size

\usepackage{fontspec} % For loading fonts
\defaultfontfeatures{Mapping=tex-text}
 % Main document font

\usepackage{xunicode,xltxtra,url,parskip} % Formatting packages

\usepackage[usenames,dvipsnames]{xcolor} % Required for specifying custom colors

%\usepackage{fullpage} % Margin formatting of the A4 page, an alternative to layaureo can be \usepackage{fullpage}
% To reduce the height of the top margin uncomment: \addtolength{\voffset}{-1.3cm}

\usepackage{hyperref} % Required for adding links	and customizing them
\definecolor{linkcolour}{rgb}{0,0.2,0.6} % Link color
\hypersetup{colorlinks,breaklinks,urlcolor=linkcolour,linkcolor=linkcolour} % Set link colors throughout the document

%\usepackage{titlesec} % Used to customize the \section command
%\titleformat{\section}{\Large\scshape\raggedright}{}{0em}{}[\titlerule] % Text formatting of sections
%\titlespacing{\section}{0pt}{3pt}{3pt} % Spacing around sections

\begin{document}

\pagestyle{empty} % Removes page numbering

\font\fb=''[cmr10]'' % Change the font of the \LaTeX command under the skills section

%----------------------------------------------------------------------------------------
%	NAME AND CONTACT INFORMATION
%----------------------------------------------------------------------------------------

\par{\centering{\Huge \textsc{Esteban Foronda Sierra}}\bigskip\par} % Your name

\section{Personal Data}

\begin{tabular}{rl}
\textsc{Place and Date of Birth:} & Colombia | November 2nd 1994\\
\textsc{Phone:} & (+57) 321 835 0291\\
\textsc{email:} & \href{mailto:seforonda@gmail.com}{seforonda@gmail.com}\\
\textsc{Linkedin:} & \href{https://www.linkedin.com/in/esteban-foronda-sierra-217446b2}{https://www.linkedin.com/in/esteban-foronda-sierra-217446b2}\\
\textsc{Github:} & \href{https://github.com/EstebanFS}{https://github.com/EstebanFS}\\
\end{tabular}


%----------------------------------------------------------------------------------------
%	Interests / Hobbies
%----------------------------------------------------------------------------------------

\section{Interests / Hobbies}
\begin{tabular}{r|p{11cm}}
& \footnotesize{Coding, Teaching, Programming Challenges, Competitive Programming Contests, Learning new things, Science.}
\end{tabular}

%----------------------------------------------------------------------------------------
%	About Me
%----------------------------------------------------------------------------------------
\iffalse
\section{About Me}
\begin{tabular}{r|p{11cm}}
& \footnotesize{I'm just a curious person that really loves to learn new things. I like to ask myself how everything work and how things could be improved, because I think nothing is in a final state, everything can be evolved in a new wonderful thing with new features.
\newline
\newline
I love the way is possible to mix science and technology to make things faster and better, that's why I would love to find a way to contribute from my knowledge to this area.
\newline
\newline
I'm a person that loves what he does, I strongly agree with the idea that the only way to be happy is doing what you love, doesn't matter how much money you could earn doing another thing, how comfortable you could be right now, or simply how hard it could be get it, you have to try it and keep looking forward to it.
\newline
\newline
I also love work in a team, because, I think each person can learn amazing things from each other, and I'm not talking just in a professional way, everybody has a lot to teach to you if you are open enough to see it and receive it.
\newline
\newline
Finally I love to teach, I think seeing people interested by something trying to improve his knowledge and reaching his goals is wonderful, that's why I love it, specially when a person have a truly intention to learn. I strongly agree with the idea that knowledge is not something you can restrict and share it is the best way to make society better.}
\end{tabular}
\fi


%----------------------------------------------------------------------------------------
%	EDUCATION
%----------------------------------------------------------------------------------------

\section{Education}
\begin{tabular}{rl}
\textbf{Current studies}: & \textsc{Systems Engineering, Computer Science}\\
\textbf{University:}&\textbf{EAFIT University}(Medell\'{i}n, Colombia)\\
\textbf{Graduation date:}& 2012-2016\\
\textbf{Current Cumulative (GPA)}: & 4.40(5.0 scale)\\
\iffalse
\textbf{Computer Science Subjects}:\\
&\textbf{}Programming Fundamentals.\\
&\textbf{}Principles of Software Development.\\
&\textbf{}Programming Languages.\\
&\textbf{}Data Structures and Algorithms I and II.\\
&\textbf{}Databases.\\
&\textbf{}Digital Electronics and Circuits, Digital Logic and Microcontrollers.\\
&\textbf{}Formal Languages and Compilers.\\
&\textbf{}Software Engineering.\\
&\textbf{}Technology Integration Project I and II\\
&\textbf{}Systemic Thinking, Information Systems.\\
&\textbf{}Telematics.\\
&\textbf{}Computer Graphics.\\
&\textbf{}Computer Architecture.\\
&\textbf{}Numerical Methods.\\
&\textbf{}Special Topics in Telematics.\\
&\textbf{}Special Topics in Software Development.\\
&\textbf{}Special Topics in Information Systems.\\
&\textbf{}Operating Systems\\
&\textbf{}Programming Paradigms\\
&\textbf{}Management of Informatics Projects\\
&\textbf{}TCP / IP Networks\\
&\textbf{}LAN Networks\\
&\textbf{}WAN Networks\\
&\textbf{}TI Architecture\\
\textbf{Mathematics Subjects}:\\
&\textbf{}Calculus.\\
&\textbf{}Predicate and Boolean Logic.\\
&\textbf{}Discrete Mathematics, Linear Algebra.\\
&\textbf{}Statistics.\\
&\textbf{}Quantitative Methods.\\
&\textbf{}Physics.\\
\fi
\textbf{Organizations}: &\textbf{}Competitive Programming Seminar.
\end{tabular}

%----------------------------------------------------------------------------------------
%	Skills and Tools
%----------------------------------------------------------------------------------------

\section{Skills and Tools}
\begin{itemize}
 \item Proficient with \textbf{Algorithmic thinking and problem solving.}
 \item Most experience using: \textbf{C++, Java (Desktop and Android).}
 \item Most experience using: \textbf{Git version control.}
 \item Most experience using: \textbf{Scrum} like \textbf{Agile Methodology.}
  \item Moderate experience creating {\sl \textbf{Rest Web Services}} using {\sl \textbf{Spring.}}
 \item Moderate experience with database programming and design using: {\sl \textbf{SQL}}.
 \item Moderate experience with web programming and design using: {\sl \textbf{HTML, CSS and JavaScipt}}.
 \item Moderate experience with {\sl \textbf{Continuous Integration}} and {\sl \textbf{Code Quality}} using: {\sl \textbf{Jenkins and SonarQube}}.
 \item Basic experience with Functional Testing using: {\sl \textbf{Calabash and Espresso android.}}
 \item Basic experience with mobile programming and design using: {\sl \textbf{iOS with Swift}}.
 \item Basic experience with web programming and design using: {\sl \textbf{Ruby on Rails}}.
 \item Basic experience with unit tests using: {\sl \textbf{JUnit}}.
 \item Basic experience with distributed computing using: {\sl \textbf{MPI}}.
 \item Operating Systems: \textbf{Windows, Ubuntu, Mac OS.}
 \item Learning for university projects: \textbf{Assembler.}
\end{itemize}

%----------------------------------------------------------------------------------------
%	WORK EXPERIENCE
%----------------------------------------------------------------------------------------

\section{Work Experience}

\begin{tabular}{r|p{11cm}}
%------------------------------------------------

\textsc{2015 - Present} & Mobile Development Area at Ceiba Software.\\
& \footnotesize{
\begin{enumerate}
\item \textbf{EMI - Health Software:} My main work on that was to fix features in both Mobile and Back-End in order to improve response time, add customer value and centralize data by consuming and creating web services. \textbf{I was able to reduce the numbers of errors produced by management of data in around 90\%}
\begin{itemize}
\item \textbf{Mobile Technology:} Titanium - Appcelerator Android version.
\item \textbf{Back-End:} Java.
\item \textbf{Front-End:} JSP.
\end{itemize}
\item \textbf{Intern Projects:} Mainly I created generic components that make easier to mix technologies or work with some specific ones in order to improve development time.
\begin{itemize}
\item \textbf{Mobile Technologies:} Android, Xamarin.
\item \textbf{WebServices:} SprintBoot.
\item \textbf{Desktop:} Shell Scripting.
\end{itemize}
\item \textbf{Rapicredit:} I've been working focused in Back-End and Front-End, trying to optimize some critical features in the application like response time, users interactions and cleaning DB. For this task I've been working with Back-End components like \textbf{LDAP and WSO2}, also using some others programming languages like \textbf{C++ and Java} to create automated tasks. \textbf{I was able to clean the current database in around 40,000 registries with an automatic task created with C++ and Java.}
\begin{itemize}
\item \textbf{Mobile technology:} Android.
\item \textbf{Back-End} Java, Sprint, LDAP, WSO2.
\item \textbf{Front-End} JSP, CSS, HTML.
\end{itemize}
\end{enumerate}} \\
\multicolumn{2}{c}{} \\
\textsc{2013 - 2014} & Mobile Development Team of Duing Application\\
& \footnotesize{Duing is a social application for mobile devices.
My team developed the \textbf{Android} native application for Duing, I focused mainly in consume data from Web Services and management of mobile databases but also I worked with graphic interface.}\\
\multicolumn{2}{c}{} \\
\end{tabular}



%----------------------------------------------------------------------------------------
%	Projects
%----------------------------------------------------------------------------------------

\section{Projects}
\begin{enumerate}
\item \textbf{Programming Lab}: Software developed in C++ that receive a series of tasks descriptions in YAML format and a series of pipes that connect some specific jobs. The software will try to execute the connected jobs and it will show if the task could be completed along with its result, doing the same task as if you run the command \textbf{\texttt{echo hello world | wc \&}} in the terminal for example.
\item \textbf{Nower}: Application to recommend special offers based in the user location, the final user(Buyer) application was created in Android, Client (Stores) was created with \textbf{HTML5, Javascript and AngularJs} and the BackEnd was created with \textbf{Ruby on Rails.}
\item \textbf{Numerical}: An \textbf{Android} application that interpret numerical functions and evaluate them with different numerical methods learned in the subject Numerical Methods.
\item \textbf{Graphics}: \textbf{Java} desktop application with different graphics methods to represent objects and change his properties(scale, rotate, etc).
\item \textbf{CoffeeStart}: CoffeeShop Cashier Application developed in \textbf{Assembler.}
\item \textbf{A queue manager for Android} That helps to request turns with your cell phone.
\item \textbf{Checkers}:  Game of Checkers created with \textbf{Java} using \textbf{Max/Min algorithm.}
\item \textbf{BlackJack}: Game of BlackJack created with \textbf{Java.}
\item \textbf{NOTE}: You can find some of those projects on my \href{https://github.com/EstebanFS}{\textbf{Github!}}
\end{enumerate}


%----------------------------------------------------------------------------------------
%	LANGUAGES
%----------------------------------------------------------------------------------------

\section{Languages}

\begin{tabular}{rl}
\textsc{Spanish:} & Native (Colombia)\\
\textsc{English:} & Advanced.\\
\end{tabular}

%----------------------------------------------------------------------------------------
%	EXTRA-CURRICULAR ACTIVITIES
%----------------------------------------------------------------------------------------

\section{Extra-Curricular Activities}
\begin{itemize}
\item \textbf{Competitive Programming Seminar - EAFIT University}\\
\textit{Lecturer Assistant}\\
\textit{January 2014 - Present}\\
I am one of the \texttt{Competitive Programming Seminar} coordinators in EAFIT University. My work is to explain and teach some topics, problems, and solutions. Those topics include graphs, dynamic programming, ad hoc, greedy algorithms, etc, in order to help university teams to improve theirs abilities in a real programming contests.\\
Also I help introducing new people at programming world.
\item \textbf{Course ``Data Structures and Algorithms II'' - EAFIT University}\\
\textit{Teacher Assistant:}\\
\textit{January 2015 - June 2015 (Six Months)}\\
My main aim was to help students to understand many algorithms and topics, explain why those topics are really important in the design of algorithms and how to pass theory into practice. Topics include graphs, complexity, search of patterns, dynamic programming, optimization, etc.\\
\end{itemize}

%----------------------------------------------------------------------------------------
%	EXTRA-CURRICULAR COURSES
%----------------------------------------------------------------------------------------
\section{Extra-Curricular Courses}
\begin{itemize}
\item \textbf{Data Structures and Algorithms - Coursera Specialization}
\begin{itemize}
\item \textbf{Algorithmic Toolbox}\\
\textbf{State:} Completed\\
\textbf{Chosen Programming Language:} C++
\iffalse
\textbf{Summary:} Introduction to different kind of algorithms (Divide and Conquer, Dynamic Programming, Greedy Algorithms, etc). The idea of the course is to provide the students with tools to affront different kind of problems and the best way to solve them.\\
\fi
\item \textbf{Data Structures}\\
\textbf{State:} Completed\\
\textbf{Chosen Programming Language:} C++
\iffalse
\textbf{Summary:} This course gives a definition of many common data structures, including Arrays, Stacks, Trees, Dynamic Arrays, Priority Queues, Disjoint Sets, Binary Search Trees (Search Trees, AVL Trees, Splays Trees). Also, this course gives a good explanation about how those structures work and are implemented in the most common programming languages.\\
\fi
\item \textbf{Algorithms on Graphs}\\
\textbf{State:} Completed\\
\textbf{Chosen Programming Language:} C++
\iffalse
\textbf{Summary:} Explanation about graphs and their properties, starting from how is the best way to represent them taking into account the number of nodes and edges, going through how to traverse a graph, find a shortest path and minimum spanning tree. Time and space complexity are the main factor of the described algorithms.
\fi
\item \textbf{Algorithms on Strings}\\
\textbf{State:} Completed\\
\textbf{Chosen Programming Language:} C++
\iffalse
\textbf{Summary:} This course is oriented to Science and how to use string algorithms to solve many genetic problems. In this course one does see topics like Suffix Tree, Suffix Array, KMP and Burrows-Wheeler Transformation. Each module of the course give the complexity and how the algorithms were evolving from some previous one in order to improve time and memory.
\fi
\item \textbf{Advanced Algorithms and Complexity}\\
In progress
\item \textbf{Assembling Genomes and Finding Disease-Causing Mutations - Final Project}\\
In progress
\item You can find more information about this course in the following link : \href{https://www.coursera.org/specializations/data-structures-algorithms}{\textbf{Data Structures And Algorithms Specialization}}\\
\end{itemize}
\item \textbf{Certified Scrum Developer (CSD)}\\
\textbf{State:} Completed\\
\textbf{Dates:} September 2015 - September 2017\\
\textbf{Certification Authority:} Scrum Alliance\\
\item \textbf{Developing Android Apps - Udacity}\\
\textbf{State:} Free version completed
\begin{itemize}
\item You can find more information about this course in the following link \href{https://www.udacity.com/course/developing-android-apps--ud853}{\textbf{Developing Android Apps}}
\end{itemize}
\item You can find those certificates on my \href{https://www.linkedin.com/in/esteban-foronda-sierra-217446b2}{\textbf{Linkedin}}\\
\end{itemize}



%----------------------------------------------------------------------------------------
%	Contests
%----------------------------------------------------------------------------------------
\newpage
\section{Contests}
\begin{itemize}
 \item \textbf{Thirty-Forth Place} at the 2012 Colombian Programming Contest ACIS REDIS.
 \item \textbf{Tenth Place} at the 2014 Colombian Programming Contest ACIS REDIS.
 \item \textbf{Tenth Place} at the 2015 Colombian Programming Contest ACIS REDIS.
 \item \textbf{Fifth Place} at the 2016 Colombian Programming Contest ACIS REDIS.
 \item \textbf{Twentieth Place} at the ACM-ICPC South American-North Regionals, November 2014.
 \item \textbf{Eleventh Place} at the ACM-ICPC South American-North Regionals, November 2015.
 \item \textbf{Fourteenth Place} at the ACM-ICPC South American-North Regionals, November 2016.
 \item Participated in the \textbf{Google Code Jam} 2013, 2014, 2015 and 2016.
 \item Participated in the \textbf{Facebook Hacker Cup} 2015.
 \item Participated in several test contest made by local organizers.
 \item Solving different kind of algorithmic problems in several online judges such as
\href{http://uhunt.felix-halim.net/id/152728}{\textbf{UVa}}, \href{http://codeforces.com/profile/EstebanFS}{\textbf{Codeforces}} and \href{https://www.hackerrank.com/EstebanFS}{\textbf{HackerRank.}}, etc.
User for each Judge: \textbf{EstebanFS}.
\item You can find some of those solutions in Competitive Programming repository on my \href{https://github.com/EstebanFS/Competitive-Programming}{\textbf{Github!}}
\end{itemize}

%----------------------------------------------------------------------------------------
%	Achievements - School
%----------------------------------------------------------------------------------------

\section{Achievements - School}
\begin{itemize}
\item \textbf{Best Graduation Project, Neosistemas 2011}:\\
I won with ``Puzzle'', a Contable Software, based in the problem of a little company.
\item \textbf{Distincted Graduates, Neosistemas, 2011}\\
Distinction with honors among graduates of my year
\end{itemize}

\end{document}