%%%%%%%%%%%%%%%%%%%%%%%%%%%%%%%%%%%%%%%%%
% Plasmati Graduate CV
% LaTeX Template
% Version 1.0 (24/3/13)
%
% This template has been downloaded from:
% http://www.LaTeXTemplates.com
%
% Original author:
% Alessandro Plasmati (alessandro.plasmati@gmail.com)
%
% License:
% CC BY-NC-SA 3.0 (http://creativecommons.org/licenses/by-nc-sa/3.0/)
%
% Important note:
% This template needs to be compiled with XeLaTeX.
% The main document font is called Fontin and can be downloaded for free
% from here: http://www.exljbris.com/fontin.html
%
%%%%%%%%%%%%%%%%%%%%%%%%%%%%%%%%%%%%%%%%%

%----------------------------------------------------------------------------------------
%	PACKAGES AND OTHER DOCUMENT CONFIGURATIONS
%----------------------------------------------------------------------------------------

\documentclass[a4paper,10pt]{article} % Default font size and paper size

\usepackage{fontspec} % For loading fonts
\defaultfontfeatures{Mapping=tex-text}
 % Main document font

\usepackage{xunicode,xltxtra,url,parskip} % Formatting packages

\usepackage[usenames,dvipsnames]{xcolor} % Required for specifying custom colors

%\usepackage{fullpage} % Margin formatting of the A4 page, an alternative to layaureo can be \usepackage{fullpage}
% To reduce the height of the top margin uncomment: \addtolength{\voffset}{-1.3cm}

\usepackage{hyperref} % Required for adding links	and customizing them
\definecolor{linkcolour}{rgb}{0,0.2,0.6} % Link color
\hypersetup{colorlinks,breaklinks,urlcolor=linkcolour,linkcolor=linkcolour} % Set link colors throughout the document

%\usepackage{titlesec} % Used to customize the \section command
%\titleformat{\section}{\Large\scshape\raggedright}{}{0em}{}[\titlerule] % Text formatting of sections
%\titlespacing{\section}{0pt}{3pt}{3pt} % Spacing around sections

\begin{document}

\pagestyle{empty} % Removes page numbering

\font\fb=''[cmr10]'' % Change the font of the \LaTeX command under the skills section

%----------------------------------------------------------------------------------------
%	NAME AND CONTACT INFORMATION
%----------------------------------------------------------------------------------------

\par{\centering{\Huge Esteban \textsc{Foronda Sierra}}\bigskip\par} % Your name

\section{Personal Data}

\begin{tabular}{rl}
\textsc{Place and Date of Birth:} & Colombia | November 2nd 1994\\
\textsc{Phone:} & (+57) 321 835 0291\\
\textsc{email:} & \href{mailto:seforonda@gmail.com}{seforonda@gmail.com}\\
\textsc{Linkedin:} & \href{https://www.linkedin.com/in/esteban-foronda-sierra-217446b2}{https://www.linkedin.com/in/esteban-foronda-sierra-217446b2}\\
\textsc{Github:} & \href{https://github.com/EstebanFS}{https://github.com/EstebanFS}\\
\end{tabular}


%----------------------------------------------------------------------------------------
%	Interests / Hobbies
%----------------------------------------------------------------------------------------

\section{Interests / Hobbies}
\begin{tabular}{r|p{11cm}}
& \footnotesize{Coding, Teaching, Programming Challenges, Competitive Programming Contests, Learning new things, Science.}
\end{tabular}

%----------------------------------------------------------------------------------------
%	About Me
%----------------------------------------------------------------------------------------

\section{About Me}
\begin{tabular}{rl}
On progress.
\end{tabular}


%----------------------------------------------------------------------------------------
%	EDUCATION
%----------------------------------------------------------------------------------------

\section{Education}
\begin{tabular}{rl}
\textbf{Current studies}:\\
&\textsc{}Systems Engineering, Computer Science\\
&\textbf{EAFIT University}(Medellin, Colombia)\\
& 2012-2016 (Expected)\\
\textbf{Current Cumulative (GPA)}:\\
&\textsc{}4.39(5.0 scale)\\
\textbf{Computer Science Subjects}:\\
&\textbf{}Programming Fundamentals.\\
&\textbf{}Principles of Software Development.\\
&\textbf{}Programming Languages.\\
&\textbf{}Data Structures and Algorithms I and II.\\
&\textbf{}Databases.\\
&\textbf{}Digital Electronics and Circuits, Digital Logic and Microcontrollers.\\
&\textbf{}Formal Languages and Compilers.\\
&\textbf{}Software Engineering.\\
&\textbf{}Technology Integration Project I and II\\
&\textbf{}Systemic Thinking, Information Systems.\\
&\textbf{}Telematics.\\
&\textbf{}Computer Graphics.\\
&\textbf{}Computer Architecture.\\
&\textbf{}Numerical Methods.\\
&\textbf{}Special Topics in Telematics.\\
&\textbf{}Special Topics in Software Development.\\
&\textbf{}Special Topics in Information Systems.\\
&\textbf{}Operating Systems\\
&\textbf{}Programming Paradigms\\
&\textbf{}Management of Informatics Projects\\
&\textbf{}TCP / IP Networks\\
&\textbf{}LAN Networks\\
&\textbf{}WAN Networks\\
&\textbf{}TI Architecture\\
\textbf{Mathematics Subjects}:\\
&\textbf{}Calculus.\\
&\textbf{}Predicate and Boolean Logic.\\
&\textbf{}Discrete Mathematics, Linear Algebra.\\
&\textbf{}Statistics.\\
&\textbf{}Quantitative Methods.\\
&\textbf{}Physics.\\
\textbf{Organizations}:\\
&\textbf{}Competitive Programming Seminar.
\end{tabular}
\newpage

%----------------------------------------------------------------------------------------
%	Skills and Tools
%----------------------------------------------------------------------------------------

\section{Skills and Tools}
\begin{itemize}
 \item Proficient with \textbf{Algorithmic thinking and problem solving.}
 \item Most experience using: \textbf{C++, Java (Desktop and Android).}
 \item Moderate experience with database programming and design using: {\sl \textbf{SQL}}.
 \item Moderate experience with web programming and design using: {\sl \textbf{HTML, JavaScipt}}.
 \item Moderate experience with continuous integration using: {\sl \textbf{Jenkins}}.
 \item Moderate experience with code quality analyzers using: {\sl \textbf{SonarQube}}.
 \item Basic experience with Functional Testing using: {\sl \textbf{Calabash}}
 \item Basic experience with mobile programming and design using: {\sl \textbf{iOS with Swift}}.
 \item Basic experience with web programming and design using: {\sl \textbf{Ruby on Rails}}.
 \item Basic experience with unit tests using: {\sl \textbf{JUnit, XCTest}}.
 \item Basic experience with distributed computing using: {\sl \textbf{MPI}}.
 \item Tools I feel comfortable with: \textbf{Git Version Control, Atom text editor, SVN Version Control.}
 \item Operating Systems: \textbf{Windows, Ubuntu, Mac OS.}
 \item Learning for university projects: \textbf{Assembler.}
\end{itemize}

%----------------------------------------------------------------------------------------
%	WORK EXPERIENCE
%----------------------------------------------------------------------------------------

\section{Work Experience}

\begin{tabular}{r|p{11cm}}
%------------------------------------------------

\textsc{2015 - Present} & Part Mobile Development Area at Ceiba Software\\
& \footnotesize{I'm part of the Mobile Development Area at Ceiba Software located in Medellin - Colombia, since I joined the company I've been working on three main projects, following described.
\begin{enumerate}
\item \textbf{EMI - Health Software:} EMI is a big company in Colombia that offers not public health services. Currently they work with a Mobile application that stores all the data generated by doctors in order to have it virtually. They had a lot of problems with that application because it didn't support all the doctors they have, so there were many architecture problems going through database model until the way the application was designed, causing a lot of data loss and services failures. Our team worked in both Back-End and Mobile application around a year and as final result we got a good application that support all the users and store data efficiently, having all the requirements the client gave us. My main work on that was to fix features in both Mobile and Back-End, in order to improve response time, add customer value, reduce mobile battery consumption, improve data storage, centralize data by consuming and creating web services, etc.
\begin{itemize}
\item \textbf{Mobile Technology:} Titanium - Appcelerator, Android version.
\item \textbf{Back-End:} Java.
\item \textbf{Front-End:} JSP.
\end{itemize}
\item \textbf{Intern Projects:} Ceiba keep looking for improve their developments quality, that's why the company invest making intern development in order to create better general solutions that help both actual and future projects. I worked in this sub - area few months, developing general components that help to reach the company goal, mainly I created components that make easier to mix technologies or work with some specific ones, for example make the development in Xamarin Studio easier. I also worked creating a generic component in Android that could be used for each application and help to make the development faster.
\begin{itemize}
\item \textbf{Mobile Technologies:} Android, Xamarin.
\item \textbf{WebServices:} SprintBoot.
\item \textbf{Desktop:} Shell Scripting.
\end{itemize}
\item \textbf{Rapicredit:} Currently I'm working with Rapicredit, an application that works with credits, I've been working focused in Back-End, trying to optimize some critical features in the application like response time, users interactions, cleaning DB, using MySQL DB and not H2 DB for WSO2, etc. For this task I've been working with Back-End components like LDAP,  WSO2 and using some programming languages like C++ and Java to make that task easier and automatic. My main goal is to give customer value to the application, having in mind aspects like improve response time, processing, memory, etc.
\begin{itemize}
\item \textbf{Mobile technology:} Android.
\item \textbf{Back-End} Java, Sprint, LDAP, WSO2, Nginx.
\end{itemize}
\end{enumerate}} \\
\multicolumn{2}{c}{} \\
\textsc{2013 - 2014} & Part Mobile Development Team of Duing Application\\
& \footnotesize{Duing is a social application for mobile devices.
My team developed the Android native application for Duing, I focused mainly in consume data from Web Services and management of mobile databases. I had to be really careful, because that application save a lot of user data so aspects like memory, energy and time were really important, all of this based in the UX principles, I also worked with graphic interface.}\\
\multicolumn{2}{c}{} \\
\end{tabular}



%----------------------------------------------------------------------------------------
%	Projects
%----------------------------------------------------------------------------------------

\section{Projects}
\begin{enumerate}
\item \textbf{Programming Lab}: Software developed in C++ that receive a series of tasks descriptions in YAML format and a series of pipes that connect some specific jobs. The software will try to execute the connected jobs and it will show if the task could be completed along with its result, doing the same task as if you run the command \texttt{echo hello world | wc \&} in the shell.
\item \textbf{Nower}: Application to recommend special offers based in the user location, the final user(Buyer) application was created in Android, Client (Stores) was created with HTML5 and Javascript with AngularJs and the BackEnd was created with Ruby on Rails.
\item \textbf{Numerical}: An Android application in order to interpret numerical functions and evaluate them with different numerical methods learned in the subject Numerical Methods.
\item \textbf{Graphics}: Java desktop application with different graphics methods to represent objects and change his properties(scale, rotate, etc).
\item \textbf{CoffeeStart}: CoffeeShop Cashier Application developed in Assembler.
\item \textbf{CloudYa}: A Web Image Hosting Application developed in HTML5, CSS3, JavaScript and PHP.
\item \textbf{A queue manager for Android} That is, request turns by mobile.
\item \textbf{Checkers}:  Game of Checkers created with Java using Max/Min algorithm.
\item \textbf{BlackJack}: Game of BlackJack created with Java.
\item \textbf{NOTE}: You can find some of those projects on my Github! \href{https://github.com/EstebanFS}{https://github.com/EstebanFS}
\end{enumerate}


%----------------------------------------------------------------------------------------
%	LANGUAGES
%----------------------------------------------------------------------------------------

\section{Languages}

\begin{tabular}{rl}
\textsc{Spanish:} & Native (Colombia)\\
\textsc{English:} & Advanced \\
\end{tabular}

%----------------------------------------------------------------------------------------
%	EXTRA-CURRICULAR ACTIVITIES
%----------------------------------------------------------------------------------------

\section{Extra-Curricular Activities}
\begin{itemize}
\item \textbf{Competitive Programming Seminar - EAFIT University}\\
\textit{Lecturer Assistant}\\
\textit{January 2014 - Present}\\
I am one of the \texttt{Competitive Programming Seminar} coordinators in EAFIT University. My work is to explain and teach some topics, problems, and solutions. Those topics include graphs, dynamic programming, ad hoc solutions, greedy algorithms, etc, in order to help university teams to improve theirs abilities in a real programming contests.\\
Also I help introducing new people at programming world.\\
\item \textbf{Course ``Data Structures and Algorithms II'' - EAFIT University}\\
\textit{Teacher Assistant:}\\
\textit{January 2015 - June 2015 (Six Months)}\\
I was the teacher assistant in the course ``Data structures and algorithms II'', my main aim was to help students to understand many algorithms and topics, explain why those topics are really important in the design of algorithms and how to pass theory into practice. Topics include graphs and their algorithms, complexity, search of patterns, basic dynamic programming, optimization, etc.\\
\end{itemize}

%----------------------------------------------------------------------------------------
%	EXTRA-CURRICULAR COURSES
%----------------------------------------------------------------------------------------
\section{Extra-Curricular Courses}
\begin{itemize}
\item \textbf{Data Structures and Algorithms - Coursera Specialization}\\
\begin{itemize}
\item \textbf{Algorithmic Toolbox}\\
\textbf{State:} Completed\\
\textbf{Chosen Programming Language:} C++\\
\textbf{Summary:} Introduction to different kind of algorithms (Divide and Conquer, Dynamic Programming, Greedy Algorithms, etc). The idea of the course is to provide the students with tools to affront different kind of problems and the best way to solve them.\\
\item \textbf{Data Structures}\\
\textbf{State:} Completed\\
\textbf{Chosen Programming Language:} C++\\
\textbf{Summary:} This course gives a definition of many common data structures, including Arrays, Stacks, Trees, Dynamic Arrays, Priority Queues, Disjoint Sets, Binary Search Trees (Search Trees, AVL Trees, Splays Trees). Also, this course gives a good explanation about how those structures work and are implemented in the most common programming languages.\\
\item \textbf{Algorithms on Graphs}\\
\textbf{State:} Completed\\
\textbf{Chosen Programming Language:} C++\\
\textbf{Summary:} Explanation about graphs and their properties, starting from how is the best way to represent them taking into account the number of nodes and edges, going through how to traverse a graph, find a shortest path and minimum spanning tree. Time and space complexity are the main factor of the described algorithms.
\item \textbf{Algorithms on Strings}\\
\textbf{State:} Completed\\
\textbf{Chosen Programming Language:} C++\\
\textbf{Summary:} This course is oriented to Science and how to use string algorithms to solve many genetic problems. In this course one does see topics like Suffix Tree, Suffix Array, KMP and Burrows-Wheeler Transformation. Each module of the course give the complexity and how the algorithms were evolving from some previous one in order to improve time and memory.
\item \textbf{Advanced Algorithms and Complexity}\\
In progress\\
\item \textbf{Assembling Genomes and Finding Disease-Causing Mutations - Final Project}\\
In progress\\
\item You can find more information about this course in the following link : \href{https://www.coursera.org/specializations/data-structures-algorithms}{Data Structures And Algorithms Specialization}
\end{itemize}
\item \textbf{Certified Scrum Developer (CSD)}\\
\textbf{State:} Completed\\
\textbf{Dates:} September 2015 - September 2017\\
\textbf{Certification Authority:} Scrum Alliance\\
\item \textbf{Developing Android Apps - Udacity}\\
\textbf{State:}�Free version completed\\
\textbf{Summary:} That course offer a Basic/Intermediate level of Android programming, specially taking into account the main principles that each Android Programmer must know, those principles are described in the next lines.
\begin{itemize}
\item \textbf{Views}: This course gives a good explanation about views and their main features.\\
\item \textbf{Operating System} Features of Android operating system is one of the most important factors, this course explain about battery, network, threads, etc. Features that you have to keep in mind if you want to create a good Android application.\\
\item \textbf{Use of Web Services:} Currently all kind of applications need to get data from a web service, that's why this task is so important, in this course you learned about different ways to consume services, starting from AsynTask to Intent services.
\item \textbf{Data Storage:} Store data is as important as consume services, that's why you have to take special care in how and when you can store some user information. This course gives you tools to take a decision about that, stating from your own SQL administrator class to Content provider.
\end{itemize}
\item You can find those certificates on my \textbf{Linkedin} \href{https://www.linkedin.com/in/esteban-foronda-sierra-217446b2}{https://www.linkedin.com/in/esteban-foronda-sierra-217446b2}\\
\end{itemize}



%----------------------------------------------------------------------------------------
%	Contests
%----------------------------------------------------------------------------------------

\section{Contests}
\begin{itemize}
 \item Participated at the 2012 Colombian Programming Contest ACIS REDIS.
 \item Participated at the 2014 Colombian Programming Contest ACIS REDIS.
 \item Participated at the 2015 Colombian Programming Contest ACIS REDIS.
 \item Participant at the 2016 Colombian Programming Contest ACIS REDIS.
 \item Participated at the ACM-ICPC South American-North Regionals, November 2014.
 \item Participated at the ACM-ICPC South American-North Regionals, November 2015.
 \item Participated in the Google Code Jam 2013, 2014, 2015 and 2016.
 \item Participated in the Facebook Hacker Cup 2015.
 \item Participated in several test contest made by local organizers.
 \item Solving different kind of algorithmic problems in several online judges, such as
UVa, Codeforces, COJ.
handle each Judge: \textbf{EstebanFS}.
\item You can find some of those solutions in Competitive Programming repository on my Github! \href{https://github.com/EstebanFS/Competitive-Programming}{https://github.com/EstebanFS/Competitive-Programming}
\end{itemize}

%----------------------------------------------------------------------------------------
%	Achievements - School
%----------------------------------------------------------------------------------------

\section{Achievements - School}
\begin{itemize}
\item \textbf{Best Graduation Project, Neosistemas 2011}:\\
I won with ``Puzzle'', a Contable Software, based in the problem of a little company.
\item \textbf{Distincted Graduates, Neosistemas, 2011}\\
Distinction with honors among graduates of my year
\end{itemize}

\end{document}
