%%%%%%%%%%%%%%%%%%%%%%%%%%%%%%%%%%%%%%%%%
% Plasmati Graduate CV
% LaTeX Template
% Version 1.0 (24/3/13)
%
% This template has been downloaded from:
% http://www.LaTeXTemplates.com
%
% Original author:
% Alessandro Plasmati (alessandro.plasmati@gmail.com)
%
% License:
% CC BY-NC-SA 3.0 (http://creativecommons.org/licenses/by-nc-sa/3.0/)
%
% Important note:
% This template needs to be compiled with XeLaTeX.
% The main document font is called Fontin and can be downloaded for free
% from here: http://www.exljbris.com/fontin.html
%
%%%%%%%%%%%%%%%%%%%%%%%%%%%%%%%%%%%%%%%%%

%----------------------------------------------------------------------------------------
%	PACKAGES AND OTHER DOCUMENT CONFIGURATIONS
%----------------------------------------------------------------------------------------

\documentclass[a4paper,10pt]{article} % Default font size and paper size

\usepackage{fontspec} % For loading fonts
\defaultfontfeatures{Mapping=tex-text}
 % Main document font

\usepackage{xunicode,xltxtra,url,parskip} % Formatting packages

\usepackage[usenames,dvipsnames]{xcolor} % Required for specifying custom colors

%\usepackage{fullpage} % Margin formatting of the A4 page, an alternative to layaureo can be \usepackage{fullpage}
% To reduce the height of the top margin uncomment: \addtolength{\voffset}{-1.3cm}

\usepackage{hyperref} % Required for adding links	and customizing them
\definecolor{linkcolour}{rgb}{0,0.2,0.6} % Link color
\hypersetup{colorlinks,breaklinks,urlcolor=linkcolour,linkcolor=linkcolour} % Set link colors throughout the document

%\usepackage{titlesec} % Used to customize the \section command
%\titleformat{\section}{\Large\scshape\raggedright}{}{0em}{}[\titlerule] % Text formatting of sections
%\titlespacing{\section}{0pt}{3pt}{3pt} % Spacing around sections

\begin{document}

\pagestyle{empty} % Removes page numbering

\font\fb=''[cmr10]'' % Change the font of the \LaTeX command under the skills section

%----------------------------------------------------------------------------------------
%	NAME AND CONTACT INFORMATION
%----------------------------------------------------------------------------------------

\par{\centering{\Huge Esteban \textsc{Foronda Sierra}}\bigskip\par} % Your name

\section{Personal Data}

\begin{tabular}{rl}
\textsc{Place and Date of Birth:} & Colombia | November 2nd 1994\\
\textsc{Phone:} & (+57) 321 835 0291\\
\textsc{email:} & \href{mailto:seforonda@gmail.com}{seforonda@gmail.com}\\
\textsc{Linkedin:} & \href{https://www.linkedin.com/in/esteban-foronda-sierra-217446b2}{https://www.linkedin.com/in/esteban-foronda-sierra-217446b2}\\
\textsc{Github:} & \href{https://github.com/EstebanFS}{https://github.com/EstebanFS}\\
\end{tabular}


%----------------------------------------------------------------------------------------
%	Interests / Hobbies
%----------------------------------------------------------------------------------------

\section{Interests / Hobbies}
\begin{tabular}{r|p{11cm}}
& \footnotesize{Coding, Teaching, Programming Challanges, Competitive Programming Contests, Learning new things.}
\end{tabular}

%----------------------------------------------------------------------------------------
%	EDUCATION
%----------------------------------------------------------------------------------------

\section{Education}
\begin{tabular}{rl}
\textbf{Current studies}:\\
&\textsc{}System engineering ,Computer Science\\
&\textbf{EAFIT University}(Medellin, Colombia)\\
& 2012-2016 (Expected)\\
\textbf{Current Cumulative (GPA)}:\\
&\textsc{}4.39(5.0 scale)\\
\textbf{Computer Science Subjects}:\\
&\textbf{}Programming Fundamentals.\\
&\textbf{}Principles of software development.\\
&\textbf{}Programming languages.\\
&\textbf{}Data structure and algorithms I and II.\\
&\textbf{}Databases.\\
&\textbf{}Digital Electronics and Circuits, Digital Logic and Microcontrollers.\\
&\textbf{}Formal languages and compilers.\\
&\textbf{}Software engineering.\\
&\textbf{}Technology Integration Project I and II\\
&\textbf{}Systemic thinking, Information systems.\\
&\textbf{}Telematics.\\
&\textbf{}Computer Graphics.\\
&\textbf{}Computer Architecture.\\
&\textbf{}Numerical Methods.\\
&\textbf{}Special Topics in Telematics.\\
&\textbf{}Special Topics in Software Development.\\
&\textbf{}Special Topics in Information Systems.\\
&\textbf{}Operating Systems\\
&\textbf{}Programming paradigms\\
&\textbf{}Operating Systems\\
&\textbf{}Management of informatic projects\\
&\textbf{}TCP / IP Networks\\
&\textbf{}LAN Networks\\
&\textbf{}WAN Networks\\
&\textbf{}TI Architecture\\
\textbf{Mathematics Subjects}:\\
&\textbf{}Calculus.\\
&\textbf{}Predicate and boolean logic.\\
&\textbf{}Discrete mathematics, Linear algebra.\\
&\textbf{}Statistics.\\
&\textbf{}Quantitative Methods.\\
&\textbf{}Physics.\\
\textbf{Organizations}:\\
&\textbf{}Competitive Programming Seminar.
\end{tabular}
\newpage

%----------------------------------------------------------------------------------------
%	Skills and Tools
%----------------------------------------------------------------------------------------

\section{Skills and Tools}
\begin{itemize}
 \item Proficient with algorithmic thinking and problem solving
 \item Most experience using: C++, Java (Desktop and Android),
 \item Moderate experience with database programming and design using: {\sl MySQL}
 \item Moderate experience with web programming and design using: {\sl HTML, JavaScipt}.
 \item Moderate experience with continuous integration using: {\sl Jenkins}.
 \item Moderate experience with code quality analyzers using: {\sl SonarQube}.
 \item Basic experience with mobile programming and design using: {\sl iOS with Swift}.
 \item Basic experience with web programming and design using: {\sl Ruby on Rails}.
 \item Basic experience with unit tests using: {\sl JUnit, AndroidTestCase, XCTest}.
 \item Basic experience with distributed computing using: {\sl MPI}.
 \item Tools I feel comfortable with: Git Version Control, Atom text editor, SVN Version Control.
 \item Operating Systems: Windows, Xubuntu, Mac OS.
 \item Learning for university projects: Assembler.
\end{itemize}

%----------------------------------------------------------------------------------------
%	WORK EXPERIENCE
%----------------------------------------------------------------------------------------

\section{Work Experience}

\begin{tabular}{r|p{11cm}}
%------------------------------------------------

\textsc{2013 - 2014} & Part Mobile Development Team of Duing App\\
& \footnotesize{I was part of the Development of Duing App, Duing is a social application for mobile devices.
My team developed the Android native application for Duing, I focused mainly in consume data from Web Services and management of mobile databases, I had to be really careful, because that application save a lot of user data so aspects like memory, energy and time was really important, all of this based in the UX principles, also I work with graphic interface. Now Duing change some aspects, and it is called Reportt.}\\
\multicolumn{2}{c}{} \\
\end{tabular}



%----------------------------------------------------------------------------------------
%	Projects
%----------------------------------------------------------------------------------------

\section{Projects}
\begin{enumerate}
\item \textbf{BlackJack}: Game of BlackJack created with Java.
\item \textbf{Checkers}:  Game of Checkers created with Java using Max/Min algorithm.
\item \textbf{FilaFacil}: A queue manager for Android, that is, request turns by mobile.
\item \textbf{CloudYa}: A Web Image Hosting Application developed in HTML5, CSS3, JavaScript and PHP.
\item \textbf{CoffeeStart}: CoffeeShop Cashier Application developed in Assembler.
\item \textbf{Numerical}: An Android application in order to interpret numerical functions and evaluate them with different numerical methods learned in the subject Numerical Methods.
\item \textbf{Graphics}: Java desktop app with different graphics methods to represent objects and change his properties(scale, rotate, etc).
\item \textbf{Nower}: App to recommend special offers based in the user location, the final user(Buyer) app was created in Android, Client (Stores) was created with HTML5 and Javascript with AngularJs and the BackEnd was created with Ruby on Rails.
\item \textbf{Programming Lab}: Software developed in C++ that receive a series of tasks descriptions in YAML format and a series of pipes that connect some specific jobs. The software will try to execute the connected jobs and it will show if the task could be completed and his result or could not be complete, doing the same task as if you run for example "echo hello world | c \&".
\item \textbf{NOTE}: You can find some of those projects on my Github! \href{https://github.com/EstebanFS}{https://github.com/EstebanFS}
\end{enumerate}


%----------------------------------------------------------------------------------------
%	LANGUAGES
%----------------------------------------------------------------------------------------

\section{Languages}

\begin{tabular}{rl}
\textsc{Spanish:} & Native (Colombia)\\
\textsc{English:} & Advanced \\
\end{tabular}

%----------------------------------------------------------------------------------------
%	EXTRA-CURRICULAR ACTIVITIES
%----------------------------------------------------------------------------------------

\section{Extra-Curricular Activities}
\begin{itemize}
\item \textbf{Course ``Data Structures and Algorithm II'' - EAFIT University}\\
Teacher Assistant:\\
January 2015 - June 2015 (Six Months)\\
I was the teacher assistant in the course "Data structures and algorithms II", my main aim was to help students to understand many algorithms and topics, explain why those topics are really important in the design of algorithms and how to pass theory into practice. Topics include graphs and their algorithms, complexity, search of patterns, basic dynamic programming, optimization, etc.\\
\item \textbf{Competitive Programming Seminar - EAFIT University}\\
Lecturer Assistant:\\
January 2014 - Present\\
I am one Competitive Programming Seminar coordinators in EAFIT University. My work is to explain and teach some topics, problems, and solutions, that's topics include graphs, dynamic programming, ad hoc solutions, greedy algorithms, etc, in order to help teams to improve theirs abilities in a real programming contest.\\
Also I help introducing new people at programming world.
\end{itemize}

%----------------------------------------------------------------------------------------
%	EXTRA-CURRICULAR COURSES
%----------------------------------------------------------------------------------------
\section{Extra-Curricular Courses}
\begin{itemize}
\item \textbf{Data Structures and Algorithms - Coursera Specialization}\\
\begin{itemize}
\item \textbf{Algorithmic Toolbox}\\
\textbf{State:} Completed\\
\textbf{Chosen Programming Language:} C++\\
\textbf{Summary:} This course give a introduction in the different kind of algorithms(Divide and conquer, Dynamic Programming, Greedy Algorithms, etc) The idea of the course is give at the students tools to affront different kind of problems and the best way to solve it.\\
\item \textbf{Data Structures}\\
\textbf{State:} Completed\\
\textbf{Chosen Programming Language:} C++\\
\textbf{Summary:} This course give a definition of many common data structures, structures include Arrays, Stacks, Trees, Dynamic Arrays, Priority queue and Disjoint sets, Binary Search trees (Search Trees, AVL Trees, Splays Trees), also this course give a good explanation of how that structures are implemented in the most common languages and how they works.\\
\item \textbf{Algorithms on Graphs}\\
\textbf{State:} Completed\\
\textbf{Chosen Programming Language:} C++\\
\textbf{Summary:} This course give a really good explanation about graphs and their properties, starting with how is the best to represent taking care of the number of nodes and edges, going through how to traverse a graph, find a shortest path and minimun spanning tree, all of this, taking special care in time and memory complexity.
\item \textbf{Algorithms on Strings}\\
\textbf{State:} Completed\\
\textbf{Chosen Programming Language:} C++\\
\textbf{Summary:} This course is oriented to Science and how to use string algorithms to solve many genetic problem, in this course you see topics like Suffix tree, Suffix Array, KMP and Burrows-Wheeler Transformation,  each part of the course give the complexity and how the algorithms were evolving from the previous one in order to improve time and memory.
\item \textbf{Advanced Algorithms and Complexity}\\
On progress\\
\item \textbf{Assembling Genomes and Finding Disease-Causing Mutations - Final Project}\\
On progress\\
\item You can find more information about this course on the following link : \href {Data Structures And Algorithms Specialization}{https://www.coursera.org/specializations/data-structures-algorithms}
\end{itemize}
\item \textbf{Certified Scrum Developer (CSD)}\\
\textbf{State:} Completed\\
\textbf{Dates:} September 2015 - September 2017\\
\textbf{Certification Authority:} Scrum Alliance\\
\end{itemize}


%----------------------------------------------------------------------------------------
%	Contests
%----------------------------------------------------------------------------------------

\section{Contests}
\begin{itemize}
 \item Participated at the 2012 Colombian Programming Contest ACIS REDIS.
 \item Participated at the 2014 Colombian Programming Contest ACIS REDIS.
 \item Participated at the 2015 Colombian Programming Contest ACIS REDIS.
 \item Participant at the 2016 Colombian Programming Contest ACIS REDIS.
 \item Participated at the ACM-ICPC South American-North Regionals, November 2014.
 \item Participated at the ACM-ICPC South American-North Regionals, November 2015.
 \item Participated in the Google Code Jam 2013, 2014 and 2015, 2016.
 \item Participated in the Facebook Hacker Cup 2015.
 \item Participated in several test contest made by CCPL and RPC(Red de programacion competitiva).
 \item Solving diferent kind of algorithmic problems in several online judges, such as,
UVa, Codeforces, TopCoder, COJ.
handle each Judge except TopCoder: EstebanFS
handle Top Coder:estebanf01
\item You can find some of those solutions in Competitive Programming repository on my Github! \href{https://github.com/EstebanFS/Competitive-Programming}{https://github.com/EstebanFS/Competitive-Programming}
\end{itemize}

%----------------------------------------------------------------------------------------
%	Achievements - School
%----------------------------------------------------------------------------------------

\section{Achievements - School}
\begin{itemize}
\item \textbf{Best Graduation Project, Neosistemas 2011}:\\
I won with "Puzzle", a Contable Software, based in the problem of a little company.
\item \textbf{Distincted Graduates, Neosistemas, 2011}\\
Distinction with honors among graduates of my year
\end{itemize}

\end{document}
